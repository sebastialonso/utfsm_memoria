\section{Identificación del problema}

La transparencia aparece entonces como una de las más grandes amenazas a un
proyecto de filosofía \textit{Agile}. Una de sus grandes fortalezas, la aceptación y
reconocimiento de la incertidumbre inherente a la naturaleza de un proyecto de 
software, termina siendo una gran piedra en el zapato. La misma incertidumbre que
genera un énfasis de trabajo enfocado en resolver solamente lo necesario para
continuar con la siguiente etapa, produce una falta de predictibilidad que siempre
complica al presupuesto, a los tiempos reservados para el proyecto y a las
expectativas del cliente de obtener lo que necesita.

Con este contexto, se pueden identificar problemas para los siguientes actores:

\paragraph{Clientes}

Por lo general, se produce una situación en donde el jefe de proyecto y el equipo
de desarrollo son los que pueden generar una predicción, en cuanto a costos y tiempos
para nuevos requerimientos, que se ajusta mejor a la realidad que los plazos del
mismo cliente. Esto es debido a que los ciclos de desarrollo han ido dejando cada
vez más experiencia en el equipo sobre cómo gestionar los recursos para nuevas
funcionalidades. Se genera una asimetría de conocimiento de gestión y de proyección entre clientes y
equipo de desarrollo. 

\paragraph{Nursoft}

Resulta interesante para Nursoft centralizar este conocimiento con respecto a
los esfuerzos, principalmente por tres razones importantes:

\begin{itemize}
  \item Cada proyecto, una vez finalizado, se queda con toda la experiencia que generó
  en términos de costo del esfuerzo. Cualquier conocimiento derivado sobre
  la naturaleza del proyecto versus los esfuerzos invertidos en distintas 
  funcionalidades implementadas por el equipo desaparece, quedando únicamente en
  la memoria de los integrantes del equipo. 
  \item La categorización de los proyectos que son aceptados por Nursoft es 
  difícil de acotar: desde proyectos relacionados con procesos de minería, pesca,
  medicina preventiva y telemedicina hasta inversión, educación, investigación y
  manejo de conflictos internacionales, por nombrar algunos. 

  Sin embargo, Nursoft se aproxima a todos los proyectos con el mismo proceso,
  y en particular, los procesos de Planificación, Desarrollo y Entrega son 
  idénticos y homologables entre proyectos. Actualmente no se cuenta con data
  formalizada de gestión de esfuerzos para ninguna tipificación de proyecto, y
  por lo tanto, Nursoft se enfrenta a los mismos problemas, una y otra vez al iniciar un
  proyecto nuevo.
  \item Nursoft cuenta con un par de trabajos\cite{morales_2019}\cite{mas'ad_2019} que hacen,
  o pronto podrían hacer, uso de información derivada del análisis de reportes.
  Es primordial para la empresa mantener todos estos trabajos en sintonía y cercanía,
  de manera que futuros trabajos de título o trabajos, internos se conecten
  fácilmente a esta red de información.
  
\end{itemize}

\paragraph{Equipos de desarrollo}

Antes del presente trabajo, se ha intentado formalizar el proceso de la gestión
de esfuerzos dentro de los equipos, no logrando el éxito esperado por diversos
motivos. El proceso de registrar esfuerzos, no otorga ventajas
inmediatamente visibles a los miembros de los equipos, de la manera que sí lo hace
para los roles más administrativos y gerenciales. De hecho al contrario, se ha
argumentado (acerca de los procesos antiguos), que el registrar el trabajo del
día a día toma una parte, menor pero no despreciable, de la jornada diaria.

Este ánalisis de los problemas decanta en las siguientes preguntas clave para el
desarrollo de este trabajo:

\begin{itemize}
  \item ¿Cómo hacer más transparente para el cliente el proceso de desarrollo de
  software y las variables relacionadas con los esfuerzos invertidos en este proceso?
  \item ¿Cómo se traspasa el conocimiento de gestión obtenido por el equipo,
  no sólo al cliente, sino que también a la empresa?
  \item ¿Cómo se combinan exitosamente las necesidades de la gestión con una
  experiencia óptima para los equipos de trabajo?
\end{itemize}
  
