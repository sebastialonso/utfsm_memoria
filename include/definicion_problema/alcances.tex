\section{Impactos y alcances}

\subsection{Impactos}

Tomando en cuenta las preguntas planteadas en la sección Identificación del Problema,
a continuación se describe el impacto que la solución propuesta tiene en la organización:

\begin{itemize}
  \item ¿Cómo hacer el proceso de desarrollo de software más transparente al cliente?
  
  La manera de trabajar de Nursoft desde el punto de vista comercial es a través
  de planes de equipo.
  El cliente paga por un equipo de ciertas características y tamaño, lo que se
  traduce en horas efectivas de trabajo. Por ejemplo, un proyecto del último año
  contempló la construcción de una plataforma de \textit{e-learning}\footnotemark[1]
  al estilo de Coursera\cite{coursera}. El cliente contrató un plan de equipo conformado
  por un diseñador full-time, un ingeniero civil full-time, un ingeniero civil
  part-time y una hueste de otros profesionales: \textit{DevOps}\footnotemark[2], ingeniero de
  aseguramiento de calidad, arquitecto de software y asesoría comercial. Esto se
  traduce directamente a un número de horas mensuales por contrato.
  
  \footnotetext[1]{De \textit{electronic-learning}, se refiere a la educación o capacitación a través de internet.}
  \footnotetext[2]{Si bien la academia y la industria aún pelean por encontrar una definición común,
  se refiere a un conjunto de prácticas de desarrollo de software y de adminsitración de sistemas con
  el fin de levantar y automatizar infraestructura tecnológica, ambientes de desarrollo y producción, y entrega continua de valor.}
  
  Nursoft entiende como transparencia informar al cliente no sólo cuántas horas
  se inviertieron en el proyecto, sino que \textit{en qué} se inviertieron estas horas,
  con detalles que pueden ser por funcionalidad/historia de usuario o por tipo de trabajo.
  Adicionalmente, Nursoft también informa a sus clientes sobre la inversión en
  \textit{horas complementarias}\footnotemark[3].


  Con una plataforma dedicada al reporte y gestión de esfuerzos, este problema se reduce a
  agregar reportes del proyecto según filtros deseados, bajo la demanda del
  cliente y de manera automatizada.
  \item ¿Cómo se traspasa el conocimiento de gestión obtenido por el equipo, no sólo al cliente,
  sino que también a la empresa?

  El conocimiento de gestión aprendido durante la vida de un proyecto, se refleja
  directamente en los esfuerzos, y especialmente \textit{en qué tan fielmente} se registran.

  En el contexto de un proyecto particular, teniendo el agregado de esfuerzos totales
  invertidos en cierto requerimiento, permiten tanto al Cliente como al Jefe de Proyecto
  contar con información histórica y fidedigna cuando se planifiquen requerimientos similares,
  entregando inmediatamente una mejor estimación de tiempos y plazos.

  En el contexto de la empresa Nursoft, contar con un histórico de esfuerzos no sólo de historias
  de usuarios y entregas de versiones, sino que de macro-procesos\footnotemark[4] permite pronosticar y
  planificar con mayor información y eficacia variables como plazos generales de proyectos,
  equipos, capacidad de equipos (contrataciones nuevas) y oportunidades de ventas.

  Cabe además insistir que Nursoft seguirá seguirá trabajando sobre esta plataforma,
  extendiéndola a otros casos de uso y en particular sobre la mina de oro que son los esfuerzos, aplicando,
  por ejemplo, técnicas avanzadas de aprendizaje de máquinas o post-procesamiento,
  generando y descubriendo nuevos conocimientos sobre sus procesos.
  Por lo mismo, la solución propuesta debe entregar buena flexibilidad, rendimiento
  y una alta reutilización de funcionalidades. Según el trabajo de Diego Plaza\cite{plaza_2015},
  una arquitectura basada en servicios entrega los requerimientos no funcionales requeridos para la necesidad de la empresa.
  
  \footnotetext[3]{Se refiere a trabajo que no es efectivo, no es cobrable al cliente, pero crucial para una correcta gestión de un proyecto ágil: gestión, reuniones con clientes, presentaciones, \textit{sprint reviews}, \textit{daily meetings}, \textit{weekly meetings}, etc.}
  \footnotetext[4]{Los macro-procesos de Nursoft incluyen Venta, Planificación, Montaje Inicial, Desarrollo y Post-Venta.}
  
  \item ¿Cómo se combinan exitosamente las necesidades de la gestión con una
  experiencia óptima para los equipos de trabajo?
  
  Hasta este punto se tienen claras las necesidades y ventajas para la mirada
  gerencial de esta propuesta. A través de un proceso de levantamiento de necesidades
  desde los equipos, tanto desde las funcionalidades como de la UX\footnotemark[5], la
  solución propuesta apunta a ser asimilada dentro de la rutina de trabajo, no solamente
  para registrar esfuerzos sino para que entregue valor diariamente a los equipos.

  \footnotetext[5]{de \textit{\textbf{U}ser e\textbf{X}perience}, se refiere a todos los aspectos de la interacción de un usuario con un producto, servicio o una compañia.}
\end{itemize}


\subsection{Alcances}

La solución permite la posibilidad de una especie de ecosistema para aplicativos
relacionados con la gestión de esfuerzos. Sin embargo, este ecosistema potencial
escapa del contexto de este trabajo, más allá de que ya existan trabajos funcionales
en Nursoft relacionados con análisis de reportes de esfuerzos.

Las intenciones centrales del presente trabajo son primero, armar la base de dicho
ecosistema para que trabajos posteriores se alimenten de él, y segundo, entregar una
primera versión de un informe automatizado de gestión de esfuerzos para el cliente.
Nursoft se aproxima a este trabajo como un punto de partida o MVP\footnotemark[6],
puntapié inicial para un proceso iterativo incremental, cuya segunda versión aún
no termina por decidirse.

\footnotetext[6]{de \textit{\textbf{M}inimum \textbf{V}iable \textbf{P}roduct}, se refiere a la iteración mínima de un producto o servicio, que permite tanto entregar valor como validar hipótesis iniciales.}