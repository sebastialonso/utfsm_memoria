\section{Contexto}

Ágil, o \textit{agile} en inglés, se ha destacado en las últimas dos decádas por ser una 
metodología que promueve una manera diferente de pensar y actuar para gestionar 
proyectos. Desde su aparición formal en sociedad con la publicación del Maniefiesto
Ágil en 2001 hasta hoy, se ha transformado en una importante alternativa 
(la única alternativa en algunos casos) para gestionar todo tipo de proyectos,
no solamente aquellos relacionados con informática \cite{scrum_report_1}.

En particular, Agile ha tenido una recepción enorme dentro del mundo de la ingeniería
y el desarrollo de software, debido principalmente a principios que, según 
Steve Denning, miembro del directorio de Scrum Alliance, aportan con guías
sistemáticas para crear productos mejores, rápidos, convenientes y personalizados \cite{denning_2015},
en contraste a la anterior metodología preferida: cascada, o \textit{waterfall model}
en inglés.


Principios básicos de la filosofía y desarrollo ágil son \cite{beck2001agile}:

\begin{itemize}
  \item énfasis en las personas y relaciones sobre los  procesos y herramientas
  \item énfasis en software que funcione sobre una documentación amplia
  \item énfasis en colaboración activa por sobre los acuerdos en contratos
  \item énfasis en un enfoque reaccionario al cambio por la adherencia estricta 
  a un plan ya definido
\end{itemize}


Son estos principios los que han llevado a metodologías ágiles como Scrum a ser
reconocidas como filosofías exitosas\cite{scrum_report_1}.

Sin embargo, como toda metodología o decisión tomada en ingeniería, existen 
trade-offs: concesiones que uno entrega para obtener más o mejores condiciones
en otras  áreas. Y son justamente los puntos de la  derecha de los principios 
anteriormente expuestos los que más complican la gestión de un proyecto ágil.

Las principales debilidades de un filosofía ágil son la falta de buena documentación,
énfasis en funcionalidad sobre usabilidad\cite{satria_systematic}, falta de predictibilidad,
mayor esfuerzo de los miembros de los equipos de trabajo, pobre planificación de
recursos\cite{disadvantages_agile}, y una tendencia a perder de vista las necesidades
reales del proyecto por sobre necesidades ficticias.

Particularmente, los últimos tres puntos pueden re-interpretarse como una falta 
(o falla) en la transparencia. Esta falta de transparencia se presenta en la 
comunicación entre cliente y equipo, principalmente para definir las 
funcionalidades que realmente se necesitan implementar y en la estimación de 
recursos requeridos para su desarrollo. La constante respuesta al cambio de Agile
y tambien la flexibilidad que posee tienden, en el mediano y largo plazo, 
a perder el énfasis en lo más importante, el core del proyecto. 
Esto puede ocasionar una mala asignación de los esfuerzos de desarrollo, 
trabajando en funcionalidades que aportan poco valor o no son válidas con un 
usuario/cliente final. Además, un mal plan de recursos suele resultar en la toma
de malas decisiones, en el abandono de buenas prácticas, y en una pérdida en la calidad.


Estas debilidades propias de la metodología ágil han sido experimentadas de 
primera mano, por la empresa de software a medida y boutique tecnológica Nursoft \cite{nursoft}
a lo largo de múltiples de sus proyectos. El presente trabajo se realizará bajo 
el alero de esta empresa, y la solución propuesta será parte cotidiana de las labores
de sus ingenieros de software, diseñadores, jefes de proyecto y del Head of 
Technology de Nursoft.