\section{Nursoft}

\textit{Pequeña introducción contextualizando la metodología de trabajo de Nursoft, la necesidad de reporar el trabajo. 
Un repaso por los intentos anteriores desde Nursoft, para gestionar el reporte de esfuerzo,
y de generación de informes agregados de estos esfuerzos desde Nursoft hacia el cliente.}

Nursoft, a finales del año 2018, definió sus objetivos claves para el año 2019. Dentro de estos,
son particularmente relevantes los dos siguientes:

\begin{itemize}
  \item creación y estandarización de macro-procesos
  \item Ampliar la cartera de grandes clientes 
\end{itemize}


¿Por qué son relevantes estos puntos? El primero se adhiere lógicamente a lo que esta memoria planea
establecer: facilitar y alentar la práctica de registrar esfuerzos, mediante una plataforma especializada,
de la manera más eficiente pero también más lo fiel a la realidad. 
Este proceso es parte de los macro-procesos de Gestión y de Desarrollo. 

TODO: Incluir diagramas de Gestión y Desarrollo

La relevancia del segundo punto no es inmediatamente aparente, por lo que se necesita un poco de contexto.

Desde 2013 hasta antes de la implementación de esta memoria, Nursoft llevaba la gestión básica de sus
proyectos utilizando la plataforma Taiga.
Sus funcionalidades acotadas y simples hicieron un calce perfecto con el Nursoft de ese entonces,
con proyectos de menor escala, y sin muchos procesos formalizados en la parte de gestión. 
Taiga utiliza un lenguaje y una interfaz muy cómoda para desarrolladores y jefes de proyecto,
pero no para \textit{stakeholders} no-técnicos, como clientes y sus asesores.

Adicionalmente, Taiga es una herramienta muy de nicho. De hecho, según Datanyze, su porcentaje de 
porción de mercado es bajísimo, como se muestra en la siguiente tabla:

TODO agregar tabla de Datanyze

¿Qué tiene de relevancia este dato? Nursoft ha descubierto en base a clientes de grandes empresas
con los que trabajó anteriormente, el uso extendido de la plataforma Jira, coincidentemente la plataforma
que mayor cuota de mercado posee.

A mediados de año, Nursoft toma la decisión de cambiar de herramienta de gestión de proyecto a Jira,
pensando en los beneficios de conexión directa con las plataformas de los clientes (si es que las tienen)
https://www.datanyze.com/market-share/project-management/Alexa%20top%201K?page=10

\subsection{Reporte}
Donde se define la estructura de un esfuerzo, en términos de modelado de dato y se contextualiza las partes que componen a un esfuerzo.

\subsection{Herramientas de gestión de proyecto utilizadas}

\textit{Donde se explican brevemente las herramientas externas que se utilizan para llevar
la gestión del proyecto, en términos de historia de usuario, sprints, versiones, etc, y cómo se conectan con la plataforma de reportes}


\paragraph{Taiga} es un gestor de proyectos \textit{open-source}, con un set de funcionalidades
enfocadas en la simpleza y creado para startups, desarrollos ágiles y diseñadores.
Taiga fue liberado en Octubre de 2014 y la última versión fue publicada en Febrero del 2018.

Utiliza \verb|Django| y \verb|Angularjs| para su backend y su frontend, respectivamente.
Cuenta con pocas integraciones \textit{out-of-the-box}, pero soporta
\textit{webhooks}\footnotemark[7] nativamente y cuenta con una API para desarrollar 
integraciones\cite{taiga_webhooks}\cite{taiga_api}.

TODO: Mostrar un proyecto en la plataforma Taiga

\footnotetext[7]{es una estrategia para que dos aplicaciones diferentes se comuniquen, reactivamente, en tiempo real.}

\paragraph{Jira Cloud} es una herramienta propietaria y pagada de gestión de proyectos. 
Puede ser utilizada online o ser descargada e instalada localmente, y fue creada
por la corporación australiana Atlassian. Permite la gestión de proyectos en múltiples metodologías
ágiles como SCRUM, Kanban, e incluso la gestión de proyectos de otras categorías, como post-venta y
solución de errores. 

Es altamente customizable y cuenta con más de 3000 integraciones
externas \cite{jira_features}. Adicionalmente, cuenta con una API para desarrollar integraciones
personalizadas \cite{jira_api}.

TODO: Mostrar un proyecto en la plataforma Jira